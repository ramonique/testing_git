\documentclass[]{article}
\usepackage[T1]{fontenc}
\usepackage{amssymb}

 
\begin{document}



\title{Ambiguity in PMCFG}

\maketitle 

\section{PMCFG}

PMCFG definition is taken from Krasimir's thesis.  Short version here:

In PMCFG we have:
\begin{itemize}
\item $T$ -- set of terminals in the grammar
\item $N^C$ -- concrete categories
\item $F^C$ -- concrete functions
\item $(T*)^d$ -- d-tuples of strings over T 
\item f $\in F^C$ has the signature $(T*)^{d_1(f)} \times
  (T*)^{d_2(f)} \times ... \times (T*)^{d_{a(f)}} \to (T*)^{r(f)}$
\item f = ($\alpha_1, \alpha_2 ... \alpha_{r(f)}$) where $\alpha_i$ is
  a sequence of terminals and <k,l>, where $1 \leq k \leq a(f)$ and $1
  \leq l \leq d_k(f)$ --- argument of the function and field from the argument.  
\item a(f) = arity of f
\item ($f t_1 ... t_{a(f)}$) is a concrete tree of cat A if $t_i$ is
  of cat $B_i$ and there is a production $A \to f [B_1
  ... B_{a(f)}]$. If a(f) = 0, then the tree is a leaf. 
\item d(C) - dimension of a category (how many fields it should have)
\item \textbf{parse} : (A : Cat ) -> Str $\to$ Set (tree of A)
\item \textbf{lin} : (A : Cat) -> tree of A -> ($x_1, x_2, ... x_{d{A}}$)
\end{itemize}


\textbf{Assumptions on the grammar}
\begin{itemize}
\item every grammar has a start category, S
\item d(S) = 1
\end{itemize}

\section{Ambiguity}

\textbf{Total ambiguity}
The trees of category A
\begin{itemize}
\item $T_1 = f_1\ t_1^1\ t_2^1\ ...\ t_{a(f_1)}^1$ and
\item $T_2 = f_2\ t_1^2\ t_2^2\ ...\ t_{a(f_2)}^2$
\end{itemize}
generate a \textbf{total ambiguity} if \\
lin (A, $T_1$) = lin (A, $T_2$) (they are identical on all indices of
the array).

\textbf{Example}: 
\begin{itemize}
\item word senses in a multilingual dictionary (where they are
supposed to have a different translation in the other languages). 
\item {\it you}(singular, plural, polite, familiar) in the English Phrasebook.
\end{itemize}


\textbf{Idea}:
Factorize over those first, in order to avoid duplicated examples.

\vspace{10mm}

\textbf{Ambiguity}
The trees of category $A_1$, resp $A_2$ (not necessarily distinct) 
\begin{itemize}
\item $T_1 = f_1\ t_1^1\ t_2^1\ ...\ t_{a(f_1)}^1$ and
\item $T_2 = f_2\ t_1^2\ t_2^2\ ...\ t_{a(f_2)}^2$
\end{itemize}
generate an \textbf{ambiguity} if, for 
lin($A_1, T_1$) = ($x_1^1,\ x_2^1\ ...\ x_{d(A_1)}^1$) and 
lin($A_2, T_2$) = ($x_1^2,\ x_2^2\ ...\ x_{d(A_2)}^2$),
$\exists$ $1 \leq i \leq d(A_1)$ and $1 \leq j \leq d(A_2)$ with
\textbf{$x_i^1 = x_j^2$} and $\exists$ y and y' sequences of terminals
(possibly empty) such that
parse (S, y$x_i^1$y'') leads to (at least) $T_S^1$ and $T_S^2$, such
that
\begin{itemize}
\item $T_1$ is a subtree of $T_S^1$ on field i
\item $T_2$ is a subtree of $T_S^2$ on field j
\end{itemize}

i.e. $\exists$ a production in the derivation of $T_S^k$ that uses
argument i from subtree $T_1$, resp j from subtree $T_2$. 

\textbf{Remark:}
$T_1$ and $T_2$ don't necessarily need to be of the same cat, but
$T_S^1$ and $T_S^2$ are (S). See {\it duck} example ("I made her
duck").

\textbf{Discussion:}
We might need to consider 2 contexts instead of one, or change the
definition so that the trees are of the same category ?


 
\section{Contexts} 

\textbf{Remark:} We consider contexts with one hole. 

\textbf{Contexts:}
For ($T_1,T_2$) a pair of trees that generate an ambiguity, we call
the \textbf{context}, the pair ($T_S^1/T_1, T_S^2/T_2$) (trees with a
hole, corresponding to $T_1$ and $T_2$), such as
\begin{itemize}
\item lin (S, $T_S^1$) = lin (S, $T_S^2$)
\item $\exists$ ($T'_1, T'_2$) of the same cats as ($T_1,T_2$) such as 
lin ($T_S^1 [T_1 \to T'_1]$) /= lin ($T_S^2 [T_2 \to T'_2]$)
\end{itemize}
 
\vspace{10mm}

\textbf{Context Equivalence}
$T_S^1/T_1 \equiv T_S^2/T_1$ 
if the derivations of $T_S^1$ and
$T_S^2$ use the same components of $T_1$ (same indices).



\textbf{Remarks/Discussion:}
\begin{itemize}
\item there is no gender now (or other inherent parameters), because
  they would just be in a different class
\item it works for the example "min hund \"{a}r vild"/'mitt barn
  \"{a}r vilt", where {\it vilt} is only ambiguous for utrum nouns.
\item need reflection grammar, but we also need to unify the different
  equivalence classes (not done directly)
\item might need \textbf{the$\_$} functions for evaluation.
\item fits with our algorithm about generating contexts, and with the
  one about generating S-trees and then chopping down to ambiguities
  (at least I think so :P) 
\end{itemize}

\end{document} 
